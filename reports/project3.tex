\documentclass[12pt]{extarticle}
%Some packages I commonly use.
\usepackage[english]{babel}
\usepackage{graphicx}
\usepackage{framed}
\usepackage[normalem]{ulem}
\usepackage{amsmath}
\usepackage{amsthm}
\usepackage{amssymb}
\usepackage{amsfonts}
\usepackage{enumerate}
\usepackage[utf8]{inputenc}
\usepackage{listings}
\usepackage[top=1 in,bottom=1in, left=1 in, right=1 in]{geometry}
\usepackage{color}
\usepackage{float}

\definecolor{dkgreen}{rgb}{0,0.6,0}
\definecolor{gray}{rgb}{0.5,0.5,0.5}
\definecolor{mauve}{rgb}{0.58,0,0.82}

%A bunch of definitions that make my life easier
\newcommand{\matlab}{{\sc Matlab} }
\newcommand{\cvec}[1]{{\mathbf #1}}
\newcommand{\rvec}[1]{\vec{\mathbf #1}}
\newcommand{\ihat}{\hat{\textbf{\i}}}
\newcommand{\jhat}{\hat{\textbf{\j}}}
\newcommand{\khat}{\hat{\textbf{k}}}
\newcommand{\minor}{{\rm minor}}
\newcommand{\trace}{{\rm trace}}
\newcommand{\spn}{{\rm Span}}
\newcommand{\rem}{{\rm rem}}
\newcommand{\ran}{{\rm range}}
\newcommand{\range}{{\rm range}}
\newcommand{\mdiv}{{\rm div}}
\newcommand{\proj}{{\rm proj}}
\newcommand{\R}{\mathbb{R}}
\newcommand{\N}{\mathbb{N}}
\newcommand{\Q}{\mathbb{Q}}
\newcommand{\Z}{\mathbb{Z}}
\newcommand{\<}{\langle}
\renewcommand{\>}{\rangle}
\renewcommand{\emptyset}{\varnothing}
\newcommand{\attn}[1]{\textbf{#1}}
\theoremstyle{definition}
\newtheorem{theorem}{Theorem}
\newtheorem{corollary}{Corollary}
\newtheorem*{definition}{Definition}
\newtheorem*{example}{Example}
\newtheorem*{note}{Note}
\newtheorem{exercise}{Exercise}
\newcommand{\bproof}{\bigskip {\bf Proof. }}
\newcommand{\eproof}{\hfill\qedsymbol}
\newcommand{\Disp}{\displaystyle}
\newcommand{\qe}{\hfill\(\bigtriangledown\)}
\setlength{\columnseprule}{1 pt}

\lstset{frame=tb,
	language=C,
	aboveskip=3mm,
	belowskip=3mm,
	showstringspaces=false,
	columns=flexible,
	basicstyle={\small\ttfamily},
	numbers=none,
	numberstyle=\tiny\color{gray},
	keywordstyle=\color{blue},
	commentstyle=\color{dkgreen},
	stringstyle=\color{mauve},
	breaklines=true,
	breakatwhitespace=true,
	tabsize=3
}

\title{CS307 Project3: Multi-threaded Sorting Application \& Fork-Join Sorting Application}
\author{Junjie Wang 517021910093}

\begin{document}
	\maketitle 
	\section{Programming Thoughts}
	\subsection{Part1: Multi-threaded Sorting}
	In this part we are going to split a list into two halves and create sort threads to sort these two halves. Then we need to use merge thread to merge these two halves. This idea is a bit similiar to the mergeSort algorithm. Another thing worth attention is that to pass the parameters to the threads, we better define a \textbf{struct}(in this example, consists of start idx and end idx).
	\subsection{Part2: Fork-Join Sorting}
	In this part we have to implement quickSort and mergeSort algorithms using Java multi-threading APIs. Both algorithms here consist of the \textbf{dividing stage}(where we use new threads to accomplish the sorting of two halves.) and \textbf{conquering stage}(where we set a threshold, smaller than we will simply use sorting algorithm like insertionSort.)
	\section{Execution Results And Snapshots}
	The execution results are as follows:
	\begin{figure}[H]
	\centering 
	\includegraphics[width=1.0\textwidth]{../imgs/3-1.png}
	\caption{Multi-threaded Sorting}
	\end{figure} 
	\begin{figure}[H]
	\centering 
	\includegraphics[width=1.0\textwidth]{../imgs/3-2.png}
	\caption{Multi-threaded Sorting}
	\end{figure} 
	\section{Code Explanation}
	First we show the parameters passing for the first part.
	\begin{lstlisting}
	/* for multiple arguments to be passed 
	* we have to define a struct */
	typedef struct _param{
		int start;
		int end;
	} funcParam;
	...
	/* setting the parameters */
	params[0].start = 0;    params[0].end = mid; 
	pthread_attr_init(&attr[0]);
	pthread_create(&tid[0], &attr[0], sortThread, &params[0]);
	...
	/* getting the parameters by type conversion 
	   and code for the sorting thread */
	void *sortThread(void *param){
		funcParam arg = *(funcParam *)param;
		int start = arg.start, end = arg.end;
		int min, tmp, idx;
		for(int i=start;i<end;i++){
			min = in_arr[i];
			idx = i;
			for(int j=i;j<end;j++){
				if(in_arr[j] < min){
				min = in_arr[j];
				idx = j;
			}
			tmp = in_arr[i];
		in_arr[i] = in_arr[idx];
		in_arr[idx] = tmp;
		}
		}
	}
	\end{lstlisting}
	Then we show the \textbf{divide} stage of both quickSort and mergeSort algorithms in Java.
	\begin{lstlisting}
	/* divide stage for the quickSort algorithm */
	int low = begin, high = end;
	int val = array[low];
	do{
	while(low < high && array[high] >= val) high--;
	if(low < high)  { array[low] = array[high]; low++;}
	while(low < high && array[low] <= val)  low++;
	if(low < high) { array[high] = array[low];high--; }
	}while(low!=high);
	
	array[low] = val;
	
	/* from the place we put val */
	int mid = low + 1;
	quickSortTask left = new quickSortTask(begin, mid - 1, array);
	quickSortTask right = new quickSortTask(mid + 1, end, array);
	
	/* create new threads to accomplish the sorting of the two halves */
	left.fork();
	right.fork();
	
	/* wait until the threads exit */
	left.join();
	right.join();
	
	/* divide stage for the mergeSort, much simpler*/
	int mid = (begin + end) / 2;
	mergeSortTask left = new mergeSortTask(begin, mid, array);
	mergeSortTask right = new mergeSortTask(mid + 1, end, array);
	left.fork();
	right.fork();
	left.join();
	right.join();
	\end{lstlisting}
	
\end{document}